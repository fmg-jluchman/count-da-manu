\documentclass[man,donotrepeattitle]{apa7}
\usepackage[T1]{fontenc}
\title{Relative importance analysis for count regression models}
\author{Joseph N. Luchman}
\affiliation{Fors Marsh, Arlington\, VA  USA}
\authornote{ \addORCIDlink{Joseph N. Luchman}{0000-0002-8886-9717} \newline \indent Correspondence concerning this article should be addressed to Joseph N. Luchman 4250 Fairfax Drive, Suite 520  Arlington, VA 22203 Email:jluchman@forsmarsh.com Phone: 1-319-621-7109}
\leftheader{Luchman}
\shorttitle{Count Model Dominance}
\journal{Journal of Business an Psychology}
\volume{x}

\begin{document}
	
\maketitle

\section{Declarations}

Conflict of Interest: The authors declare that they have no conflict of interest.

\abstract{
	Count variables are common in organizational science as an outcome. Count regression models (CRMs), such as Poisson regression, are recommended as methods to analyze count variables but can be challenging to interpret given their non-linear functional form. I recommend relative importance analysis as a method to use in interpreting CRM results. This work extends on past research by describing an approach to determining the importance of independent variables in CRMs using dominance analysis (DA). In this manuscript, I review DA as a relative importance method, recommend an $R^2$ to use with CRM-based DA, and outline the results of an analysis with simulated data that uses the recommended methodology. This work contributes to the literature by extending DA to CRMs and provides a thoroughly documented example analysis that researchers can use to implement the methodology in their research.
}

\keywords{Dominance Analysis, Relative Importance, Poisson Regression, R-square, Negative Binomial Regression, Count Data}

\end{document}