%\documentclass[10pt,a4paper]{article}
\documentclass[ShortAfour,times,sageapa]{sagej}
\usepackage[utf8]{inputenc}
\usepackage[T1]{fontenc}
\usepackage{amsmath}
\usepackage{amssymb}
\usepackage{graphicx}


\begin{document}
	
\runninghead{Luchman}
\title{Dominance analysis for count dependent variables}
\author{Joseph N. Luchman\affilnum{1}}
\affiliation{\affilnum{1}Fors Marsh Group LLC}

\begin{abstract}
	Determining independent variable relative importance is a highly useful practice in organizational science.  Whereas techniques to determine independent variable importance are available for normally distributed and binary dependent variable models, such techniques have not been extended to count dependent variables (CDVs).  The current work extends previous research on binary and multi-category dependent variable relative importance analysis to provide a methodology for conducting relative importance analysis on CDV models using dominance analysis (DA).  Moreover, the current work provides a set of comprehensive data analytic examples that demonstrate how and when to use CDV models in a DA and the advantages general DA statistics offer in interpreting CDV model results.  Moreover, the current work outlines best practices for determining independent variable relative importance for CDVs using replaceable examples on data from the publicly available National Longitudinal Survey of Youth 1979 cohort.  The present work then contributes to the literature by using in-depth data analytic examples to outline best practices in conducting relative importance analysis for CDV models and by highlighting unique information DA results provide about CDV models.
\end{abstract}

\keywords{Dominance Analysis, Relative Importance, Poisson Regression, Negative Binomial Regression, R-square}

\maketitle

\section{Introduction}

	Organizational scientists conduct research on work-related problems that focus on many different specific topics including job performance, employee wellness, and effective task staffing.  
	Quantifying topics such as job performance often requires that researchers use data that are in the form of discrete, sometimes infrequent, events such as number of contracts won in a year, number of complaints received in a month, or number of days absent for illness in a business quarter. % use actual examples here?
	Discrete, infrequent event data, called \textit{count dependent variable models}/(\textit{CDMs}) in this paper, are useful representations of many concepts in organizational science but can present additional complications for analysis.
	One such complication is that count data can diverge from the statistical assumptions made of the Normal or Gaussian distributed linear regression model---by far one of the most common predictive models applied in organizational science \cite{}.
	
	
	CDVs are often modeled using generalized linear models adapted to the structure of infrequent events.
	Poisson or negative Binomial regressions \cite{} are commonly applied to CDMs as they tend to fit better with positive integer-valued data than do Normal/Gaussian distributions.  
	Although models such as the Poisson regression fit better to CDVs, Poisson and similar regressions are more complex to interpret than linear regression as they are intrinsically non-linear.
	In addition, CDVs' discrete, event-oriented nature requires additional considerations that tend not to apply to continuous, Normally-distributed data.
	
	
	Decision makers in industry, government, and non-profit organizations look to organizational scientists to estimate and interpret CDV models when required given the research question. 
	In linear regression models, a commonly used tool to assist in the interpretation of statistical modeling is to evaluate and compare the relative importance of the independent variables \cite{}. 
	Comparing independent variables and determining their importance relative to one another is most often accomplished using the dominance analysis/(DA) \cite{} approach. 
	Published methodological work on DA has discussed multiple intrinsically non-linear models including binary \cite{}, ordered, and multinomial logit \cite{} models but has not provided an extensive discussion of how to implement and interpret DA with CDVs.  
	Moreover, CDVs can require adjustments to modeling such as the consideration of \textit{exposure} that can affect a DA's relative importance determination results.
	
	
	The goal of this work is to extend on Blevins, Tsang, and Spain \cite{blevins2015count} in extending considerations regarding CDV models to assessing relative importance using DA...
	
	 and then offer recommended practice for applying DA to CDV models.
	This paper is organized into ... sections.  
	The first section reviews DA
	The se
		
\section{Dominance Analysis}
	
	DA is an extension of Shapley value decomposition from Cooperative Game Theory % needs caps?
	\cite{} which seeks to find a solution to the problem of how to subdivide payoffs to players in a cooperative game based on their relative contributions.
	
	
	The Shapley value decomposition method views the predictive model as a cooperative game where the different independent variables work together to predict the dependent variable.  
	The payoff from the predictive model is the value of the model fit statistic; usually this payoff is an $R^2$. % more on what the "payoff" means?
	
	
	This methodology can be applied to predictive modeling in a conceptually straightforward way.
	 Predictive models are, in a sense, a game in which independent variables cooperate to produce a payoff in the form of predicting the dependent variable. 
	The component of the decomposition/the proportion of the payoff ascribed to each independent variables can then be interpreted as the IVs importance in the context of the model as that is the contribution it makes to predicting the dependent variable.
	
	
	In application, DA determines the relative importance of IVs in a predictive model based on each IV’s contribution to an overall model fit statistic—a value that describes the entire model’s predictions on a dataset at once. 
	DA’s goal extends beyond just the decomposition of the focal model fit statistic. 
	In fact, DA produces three different results that it uses to compare the contribution each IV makes in the predictive model against the contributions attributed to each other IV. 
	The use of these three results to compare IVs is the reason DA is an extension of Shapley value decomposition.
	
	Complete dominance between two IVs is designated by:
	
	\begin{equation}
		X_{v}DX_{z}\ if\ 2^{p-2} = \Sigma{2} %\^{}\{2\^{}\{p-2\}\}\_\{j=1\}\{ \textbackslash{}\{\textbackslash{}begin\{matrix\} if\textbackslash{}, F\_\{X\_v\textbackslash{}; \textbackslash{}cup\textbackslash{};S\_j\}\textbackslash{}, > F\_\{X\_z\textbackslash{}; \textbackslash{}cup\textbackslash{};S\_j\}\textbackslash{}, \textbackslash{},then\textbackslash{}, 1\textbackslash{}, \textbackslash{}\textbackslash{} if\textbackslash{}, F\_\{X\_v\textbackslash{}; \textbackslash{}cup\textbackslash{};S\_j\}\textbackslash{}, \textbackslash{}le F\_\{X\_z\textbackslash{}; \textbackslash{}cup\textbackslash{};S\_j\}\textbackslash{},then\textbackslash{}, \textbackslash{},0\textbackslash{}end\{matrix\} \}
	\end{equation}
	
	Where $X_v$ and $X_z$ are two IVs, $S_j$ is a distinct set of the other IVs in the model not including $X_v$ and $X_z$ which can include the null set (...) with no other IVs, and $F$ is a model fit statistic. Conceptually, this computation implies that when all $2^{p-2}$ comparisons show that $X_v$ is greater than $X_z$, then $X_v$ completely dominates $X_z$.
	
	Conditional dominance statistics are computed as:
	
	\begin{equation}
		C^{i}_{X_v} =
	\end{equation}
	%CiXv=Σ[p−1i−1]i=1FXv∪Si−FSi[p−1i−1]
	
	Where $S_i$ is a subset of IVs not including $X_v$ and [p-1i-1] is the number of distinct combinations produced choosing the number of elements in the bottom value (i-1) given the number of elements in the top value (p-1; i.e., the value produced by choose(p-1, i-1)).
	
	In effect, the formula above amounts to an average of the differences between each model containing $X_v$ from the comparable model not containing it by the number of IVs in the model total.
	
	General dominance is computed as:
	
	\begin{equation}
		C_{X_v} = \frac{\sum_{p}^{i} C^{i}_{X_v}}{p}
	\end{equation}
	%CXv=Σpi=1CiXvp
	
	Where, $C^{i}_{X_v}$ are the conditional dominance statistics for $X_v$ with $i$ IVs. 
	Hence, the general dominance statistics are the arithmetic average of all the conditional dominance statistics for an IV.
	
	In the section below, I transition to discussing some of the nuances of CDVs for the application of DA.
	
\section{Applying Dominance Analysis to Count Dependent Variable Models}

	As a variant of Shapley Value Decomposition, DA is a generally applicable method for decomposing a fit statistic from a statistical model into components.
	In discussing how to apply DA to CDV models, there are several key considerations that make CDV models more complex than linear models with continuous DVs.  
	First, CDVs are generalized linear models and traditionally assume a multiplicative relationship between an IV and the DV such that a one unit change in the IV result in a change in the DV that is the anti-log or exponential function (i.e., $e^{\beta}$) of the coefficient.  
	Thus, in a CDV model, predicted counts produced by a change in the IV depends on the location of the DV.  
	For instance, a model coefficient of .5 in a linear model will always add .5 to the DV such that 
	
	For a CDV, it is common to interpret the coefficient as an incidence rate ratio 1.65
	
	will produce a change at a count of 3 that is 1.9 higher (i.e., $e^{3 + .5*1} - e$)
	
	The multiplicative nature of CDV models complicates fit metric computation in that the standard fit metric, the explained variance $R^2$ assumes that the underlying model seeks to minimize the sum of the squared residuals from the model.  As will be discussed below, CDVs do not meet this criterion.
	
	
	A second consideration relevant to CDV models is the use of model offsets.  An offset is a variable that is intended to reflect \emph{exposure} or differences between observations in the capability for that observation to produce a count.  Offset variables are usually factors such as population sizes (cite) or exposure time windows (cite) that will affect the observations' counts and are known about different observations beforehand.  Offsets are included into the model with a coefficient of 1 and serve to make the CDV a rate as they adjust the count such that $e^{y_{i} - offset_{i}} = $
	
	
	Finally, a common model applied to CDVs are \emph{zero inflated} models that are recommended for use in modeling CDVs in many situations (see Figure 3 of \cite{blevins2015count}).  Zero inflated models offer a great deal of flexibility in evaluating the processes
	
	The flexibility of zero inflated models comes at the cost of greater complexity when considering how to evaluate the contributions IVs have to prediction as there are two predictive equations, the count-producing model and the zero-producing model, which need not have the same set of predictors.  As such, it may be necessary to examine parameter estimate relative importance (PERI; \cite{luchman2020relative}) as opposed to independent variable relative importance (IVRI) when examining 
	
	
	In the sections below, each of the three complications regarding CDVs is discussed in greater detail with a focus on how each can affect the determination of relative importance.
		
	
	\subsection{Fit Metrics for Count Dependent Variables}
	
	
	CDMs are inherently multiplicative in how the underlying prediction model relates to the observed count DV. 
	The multiplicative nature of CDMs complicates assessing how well the model fits to the data as the way in which residuals are computed for the LRM does not produce useful.
	Consider the standard explained variance $R^2$.  The $R^2$ is the squared Pearson correlation between the DV and the predicted values from the model. 
	Pearson correlations, like the LRM, are based on the sums of squared deviations ...  
	
	Consider now the simplest of the CDMs, the Poisson regression model. The Poisson regression minimizes the sum of deviations from the product of the DV and the natural log of the predicted value or $\sum (\mu - y \ln \mu)$ \cite{cameron1996r}. The form of the minimizing value for the Poisson makes it is possible that using a metric like the explained variance $R^2$ that computes the sum of squared deviations will produce negative contributions to the $R^2$.
	Specifically, the explained variance $R^2$ rewards smaller total or summed squared deviations from the observations.
	By contrast, a model like the Poisson prefers minimizing deviations with a preference toward shrinking to 0; hence, penalizing more for over-prediction than under-prediction.
	The differences between these penalty factors mean that a Poisson model might choose a coefficient that shrinks predicted values to such an extent that it appears to reduce fit to the model when applying an explained variance metric.
	The more complex negative Binomial model shows similar properties and, as is noted by Spain et al., is a direct extension of the Poisson (i.e., a mixture of Poisson and Gamma distributions).
	
	A similar discussion to the one above is provided by Traxel and Azen \cite{azen2009using} focused on the closely related generalized linear model, Logistic regression.  
	Azen and Traxel show that likelihood-based pseudo-$R^2$ metrics, in particular the McFadden and Estrella $R^2$s, are useful for DA with logistic regressions as they always increase with the inclusion of more predictors and show several other desirable properties.
	
	$R^2$ recommendations specific to CDMs have also been proposed in the literature.  
	Cameron and Windmeijer \cite{cameron1996r} provide perhaps the most comprehensive review and critique of different fit metrics for CDMs.
	Cameron and Windmeijer, similar to Azen and Traxel, focus on desirable properties of $R^2$s but conclude that the $R^2$ that best reflects fit to the data for CRMs is the deviance $R^2$ or $R^{2}_{DEV}$.
	The deviance $R^2$ is computed as:
	$$R^{2}_{DEV} = \frac{D_{model}}{D_{null}}$$
	Where $D$ is the GLM deviance residual computation.
	In fact, the McFadden $R^2$ recommended by Axen and Traxel is tantamount to the deviance $R^2$ for the logistic model given the way the model is scaled, but differs for CRMs by a factor of ...
	
	In sum, the recommended $R^2$ metric is the deviance $R^2$ which has a number of desirable properties that make it similar to the explained variance $R^2$ across a number of the models discussed by Blevins, Tsang, and Spain \cite{blevins2015count}.  
	These desirable properties make $R^{2}_{DEV}$ the fit metric that is recommended for use in DA when using CRMs.  
	Further discussion of fit metrics for modeling and DA in CRMs is continued below as applied to the data analytic examples. 
	
	\subsection{Exposure and Offset Terms}
	
	CDMs, as models of counts of discrete events, implicitly assume that the events being counted are comparable.  
	An observation with a count of 10 is assumed to have had the same chance to get that count of 10 events as any other observation with 10 events.  
	That is, that the probability of observing events for both observations is equal.
	By contrast, it is possible two observations with an observed count of 10 arrived at their count with different underlying probabilities of observing events and number of times the events could be realized into a count.  
	For example, two 10 counts could have resulted from an observation with an event probability of .1 over 100 'trials'.  
	Another could have arisen from an observation with an event probability of .01 over 1000 'trials'.  
	In data, such equifinality in counts from different underlying probabilities can arise from observations that are units reflecting populations of different sizes or units reflecting different time periods in which events could occur (i.e., unequal employment tenures).
	
	CDMs can control for such unequal \emph{exposure} to the count generating conceptual phenomenon using what is known as an \emph{offset} term.  
	The offset term is usually assumed to be natural log transformed for a CDM and enters into the CDM with a coefficient of 1.
	The coefficient of 1 results in the count DV being transformed into a rate out of the offset variable.
	How the count DV is transformed into a rate with a coefficient of 1 follows from the following algebraic manipulations.
	First, consider a simple case where an intercept only model is fit with an offset such as $y = e^{offset + \beta}$.
	Recall that the offset term is the natural log of a variable, say, $o$.  
	Applying the natural log, the prior equation can be written as $\ln y = \ln o + \beta$.
	Re-arranging the $\ln o$ term results in $\ln y - \ln o = \beta$ which, given the properties of logarithms, is tantamount to $\ln \frac{y}{o} = \beta$.
	Therefore, the inclusion of a natural log transformed offset variable results in the count DV transforming into a rate.
	This offset transformation corrects for the issue discussed above as the 10 counts result in the correct underlying rate of .1 versus .01 for each observation.
	
	In large part, an offset adjustment serves to re-scale specific observations' predictions and has the biggest effect on CDM's intercept value; adjusting it back to reflect the correct average rate across observations.
	The offset can, however, affect the magnitude of estimated coefficients if the offset is correlated with an IV.  
	When the offset is correlated with an IV, then the IV may be both related to exposure to the count generating phenomenon as well as the rate of count generation broadly.
	Not including the offset can then bias the CDM and can have a notable impact on IV relative importance.
	How offsets affect CRM modeling and importance determination will also be explored further in the empirical examples below.
	
	% on to examples?
	
	\subsection{Multiple Equations}
	
	... Poisson vs alternatives ...
	
\section{Data Analytic Examples: DA with CRMs}

	% empirical examples illustrating impact of fit metric, offsets, and mutiple equations
	% also show impact of model choice/using the wrong model
	
	% make several different versions - 1) uncorrelated vars, 2) moderate, 3) strongly corrd
	% poisson, neg bin, and zip DVs for each
	% several offsets?
	
	% take from below?  Otherwise below might be cut
	
	The purpose of this manuscript is to highlight specific issues in CRMs that affect how relative importance is determined using DA.
	The three concepts discussed were chosen for their relevance to CRMs and their impact on determining DA using CRMs.
	This section extends on the discussion above with a number of data analytic examples that illustrate the concepts discussed above with generated data.
	The data analytic examples also highlight areas that might affect the DA with CRMs in ways that are less crucial but nonetheless noteworthy.
	
	\subsection{Data Generation}
	
	The data for the analytic examples were generated using a combination of base R \texttt{stats} \cite{R} and the \texttt{MASS} packages \cite{MASS}.
	In each data analytic example, four IVs were generated with different covariance structures and variances assuming multivariate normality of the underlying distributions of the predictors.
	Four total IV covariance structures were created.  The first was an equal variance, equal uncorrelated set of four IVs for use as a comparison.  The second was an equal variance power-law type decay (.5, .25, .125..) set of correlations between the predictors.  The third was an uncorrelated with unequal power-law type variance...  The final was a combination of the power-law correlation and variances.
	In all cases, the strength of the correlations generated between the variables varied inversely with the size of the variance generated for that variable % have to describe this better...
	
	

	
	
	...
	
	
	CDV models are complex, inherently multiplicative (note that it is possible to estimate them as non-multiplicative - but this is rarely done) models that require the use of estimation techniques such as maximum likelihood to obtain parameter estimates and sampling variances.
	It is helpful to discuss some of the nuances of how these models are estimated to understanding the implications of these complexities for DA.
	Hence, in the sections to come, I discuss aspects 

	\subsection{Poisson Regression: The Most Basic Case}
	
	The most conceptually simple CDV model is the Poisson regression.  
	The Poisson regression's log likelihood is a sum of three components as is shown below in (can I reference this equation?).
	
	\begin{equation}
		\ln L = \sum_{i=1}^{N} xb_{i}y_{j} - (\ln (y_{i}!) + e^{xb_{i}})
	\end{equation}

	Where $xb_{i}$ refers to a respondent's untransformed/linear predicted value from the Poisson regression.  
	Of note with this log-likelihood is the division into two separate components.  
	On the one hand, the log-likelihood increases, with the $xb_{i}y_{j}$ term.  
	Hence, the product of the observed CDV value and the predicted value for a respondent contributes directly to the log-likelihood and indicates better fit to the data.
	On the other hand, the second set of terms, $\ln (y_{i}!) + e^{xb_{i}}$ decrease the log-likelihood.  
	Thus, has the log factorial of the CDV increases and as the exponential function of the predicted values increase without a concomitant increase in the $xb_{i}y_{j}$.
	
	For example consider a respondent with a CDV value of 5.  
	The best fit to this value would be a transformed predicted value of 5 or an untransformed/linear predicted value of $e^{5} = -1.740$.  
	When applied to the log-likelihood function, the value obtained is $5*-1.740 - \ln (5) + e^{-1.740} = -1.740$ or untransformed/linear predicted value.
	By contrast, two hypothetical less ideal predicted values might be 4 and 6.  
	In both cases, the log-likelihood values diverge from the best fitting one and increase, indicating a lower likelihood.
	Specifically, 4 gives $5*\ln (4) - \ln (5) + 4 = -1.856$ and 6 gives $5*\ln (6) - \ln (4) + 6 = 1.829$...
	This also shows an important point about the Poisson log likelihood--that it is not symmetric around the best fitting value.  
	By contrast, ordinary least squares regression would penalize both 4 and 6 as squared deviations of 1 from the line of best fit (footnote that logged DV can do this too - but the loglik is still symmetrical and the model is now fitting the mean of a transformed DV as opposed to the original DV).
	That the model fit to the data is asymmetrical has implications for assessing fit to the model that will be discussed in greater detail later.
	
	\subsection{Negative Binomial: A More Complex and Flexible Case}
	
	A constraint with the Poisson model is that it has only a single parameter estimated from the data, it's mean.  The variance of the Poisson distribution is, by assumption, identical to the mean.
	
	One way to extend on the Poisson to make it more flexible is to conceptualize the mean of the Poisson distribution as a random variable.  If the mean of the Poisson is conceptualized as Gamma distributed, the hybrid Poisson-Gamma distribution can be reduced to a unique distribution called the negative Binomial.
	
	The negative Binomial model is a more complex generalization of the Poisson that relaxes the assumption that the variance and mean are identical
	
	...negative binomial...
	
	\begin{equation}
		\ln L = \sum_{i=1}^{N} \ln (\Gamma (\frac{1}{\alpha} + y_{i}) ) + m- (\ln (y_{i}!) + e^{xb_{i}})
	\end{equation}

	

	% rework below

	In applying DA to CDVs, many readers might question whether there is a need to use methods other than the standard linear regression with the explained variance $R^2$ metric.
	
	Although CDV models share a number of similarities with continuous DVs, CDVs and the models designed to work with them are a distinct subset of generalized linear models and behave differently than linear regression in ways that  \emph{(cite Blevins and summarize somewhere around here)}.
	
	One similarity across count and continuous DVs is that, as the mean of a CDV variable increases, it grows increasingly good at being approximated with a Normal distribution \emph{(cite!)}.  
	This points to an important difference when applied to CDVs with rare events in that the distributions for count and continuous DVs diverge notably when the mean is nearer 0 (example here?).  
	Such divergence results in important differences to model-to-data fit that not only affects the model fit metric's value overall but also how specific IVs explain variation in the CDV.  
	In particular, CDVs are discrete, non-negative integers and, accordingly, CDV models are discrete probability distributions that accommodate non-negative integers.  
	Normal distributions are continuous
	
	
	

	Applying DA to CDVs

	% note to self - lean on Blevins, Tsang, and Spain (2014) for this discussion...

	A useful first step toward defining recommended practice for applying DA to CDVs is to understand how CDV models differ from linear models.
	Differences between CDV and linear models affect model estimation and how relative importance among IVs should be determined.	
	One substantial difference between CDV models such as the Poisson from linear models is in the functional form of the model. 
	In a linear model, the magnitude of the regression coefficient for an IV reflects the expected change in the DV given one unit of change in the IV. 
	By contrast, CDV models most often use a log-linear linking function.
  	In a log-linear link model like the Poisson, the coefficients are estimated using from the data as though they were transformed using a natural logarithm.  This implied transformation, or linking function, results in the CDV effectively ranging over all real numbers just like a continuous DV is expected to in linear regression.
  	
  	  	% equation here?%
  	\begin{equation}
  		ll = \frac{1}{1}
  	\end{equation}
  	
  	Although the CDV is implied to range over all real numbers in the estimation algorithm, the observed CDV is not changed and the predicted values from the CDV model are in log-linear units as opposed to those of the CDV (i.e., counts of the event).  	
  	In order to produce meaningful predicted values, CDV models need to back-translate their predicted values to the metric of the CDV.  The back-translation applies an anti-log or exponential function to the predicted values.
  	 
  	the natural logarithm linking function used to translate the predicted values
  	model from a linear model back to the original count metric results in a one unit change in the IV producing a different magnitude of change to the dependent variable depending on where on the continuum of the dependent variable the change is located. 

	The log-linear nature of the coefficients produced by CDV models make the difficult to interpret directly. 
	Typically, CDV coefficients are translated using an exponential function to produce \emph{Incidence Rate Ratios} or \emph{IRRs}.  
	that naturally produce multiplicative effects across the range of each IV.

	The naturally multiplicative functional form of CDV models makes the explained-variance $R^2$ metric less useful for DA.  This is because CDVs are not guaranteed to produce an increase to the explained-variance $R^2$ as more IVs are added to the model \cite{}.  
	
	There are pseudo-$R^2$s that are better able to characterize model fit for CDVs.  
	
	
	\begin{equation}
		\ln{y} = \sum{\beta_{x_i}}
	\end{equation}


\bibliography{CountDominance.bib}
\bibliographystyle{mslapa}
	
\end{document}